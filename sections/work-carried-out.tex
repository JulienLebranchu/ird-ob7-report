%%%%%%%%%%%%%%%%%%%%%%%%%%%%%%%%%%%%%%%%%
%%% Explanation of the Work carried out
%%% by the Beneficiaries and Overview of 
%%% the Progress
%%%%%%%%%%%%%%%%%%%%%%%%%%%%%%%%%%%%%%%%%

\clearpage
\section{Explanation of the Work carried out by the Beneficiaries and Overview of the Progress}
\label{sec:work-carried-out}

%%%%%%%%%%%%%%%%%%%%%%%%%%%%%%%%%%%%%%%%%
%%% Section content, please change!
%%%%%%%%%%%%%%%%%%%%%%%%%%%%%%%%%%%%%%%%%

This section summarises the work carried out by the \erigrid2 consortium during the \nth{0} \ac{RP} towards the project goals and work plan as described in the \ac{GA} \cite{bib:grantagreement2019}. 

\subsection{Objectives}

\todo{List the specific objectives for the project as described in Section 1.1 of the \ac{DoA} and described the work carried out during the reporting period towards the achievement of each listed objective. Provide clear and measurable details.}

\subsection{Explanation of the Work Carried per \acl{WP}}

This section provides an overview of the work carried out and results achieved per \ac{WP} 
during the the \nth{0} \acl{RP}.   

\input{sections/WP01}
\input{sections/WP02}
\input{sections/WPxy}

\subsection{Impact}

\todo{Include in this section whether the information on section 2.1 of the DoA  (how your project will contribute to the expected impacts) is still relevant or needs to be updated. Include further details in the latter case.}

\subsection{Access Provisions to \aclp{RI}}

\subsubsection{\acl{TA}}

\paragraph{Description of the publicity concerning the new opportunities for access} \mbox{}

\todo{In the first periodic report describe the measures taken to publicise to research teams throughout Europe the opportunities for access open to them under the \ac{GA}. In the following periodic reports indicate only additional measures and changes.}

\paragraph{Description of the selection process} \mbox{}

\todo{In the first periodic report, describe the procedure used to select users: organisation of the Selection Panel, any additional selection criteria  employed by the Selection Panel, measures to promote equal opportunities, etc. Specify if feedback is given to rejected applicants and in which form. In the following periodic reports indicate only changes to the existing procedure.\\[0.5em]
The list of the Selection Panel members should be maintained and update when necessary in order to prove that the panel is composed following the conditions indicated in Article 16.1 of the \ac{GA}. The Commission reserves the right to request this list at any time.\\[0.5em]
Indicate number, date and venue (if not carried out remotely) of the meetings of the Selection panel during the reporting period.\\[0.5em]
Provide integrated information on the selection of user projects and on the scientific output of supported users. In particular  indicate the number of eligible User projects submitted in the reporting period and the number of the selected ones taking into account only calls for which the selection has been completed in the reporting period. Indicate also the number of user projects, started and supported in the reporting period, which have a majority of users not working in an \ac{EU} or associated country.}

\paragraph{Description of \ac{TA}} \mbox{}

\todo{Give an overview of the user-projects  and users supported in the reporting period indicating their number, their scientific fields and other relevant information you may want to highlight. You should maintain the list of the user-projects for which costs have been incurred in the reporting period. A user-project can run over more than one reporting period. In this case it should be inserted in the list of each concerned reporting period.\\[0.5em] 
The list of user-projects must include, for each user-project, the acronym, objectives, as well as the amount of access granted to it on each installation used by the user-project in the reporting period. When the user-project is completed in the reporting period the list should also include a short description of the work carried out. The Commission reserves the right to request this list at any time.\\[0.5em]
In addition you must fill the following tables (in Part A to be filled in the IT tool):
\begin{itemize}
    \item List of users:  Researchers who have access to \acp{RI}/installations (one or more) through Union support under the grant either in person (through visit) or through remote access; 
    \item \acp{RI} made accessible to all researchers in Europe and beyond through EU support and summary of trans-national access provision per installation per reporting period indicate for each installation providing trans-national access under the project the quantity of access actually provided in the Reporting Period (expressed in the unit of access defined in Annex 1 for that specific installation). 
\end{itemize}
}

\paragraph{Scientific output of the users at the facilities} \mbox{}

\todo{Give highlights of important research results from the user-projects supported under the grant agreement. Indicate the number and the type of publications derived by user-projects supported under the grant taking into account only publications that acknowledge the support of this \ac{EU} grant.\\[0.5em]
You should maintain a list of publications that have appeared in journals (or conference proceedings) during the reporting period and are resulting from work carried out under the \ac{TA} activity. List only publications that acknowledge the support of the European Community. For each publication indicate: the acronyms of the user-projects that have led to the publication itself, the authors, the title, the year of publication, the type of publication (Article in journal, Publication in conference proceeding/workshop, Book/Monograph, Chapters in book, Thesis/dissertation, whether it has been peer-reviewed or not, the \ac{DoI}, the publication references, and whether the publication is available under \ac{OA} or not. The Commission reserves the right to request this list at any time. 
}

\paragraph{User meetings} \mbox{}

\todo{If any user meetings have been organised in the reporting period, indicate for each of them the date, the venue, the number of users attending the meeting and the overall number of attendees.}

\subsubsection{\acl{VA}}

\todo{Provide statistics on the \ac{VA} in the period by each installation, including quantity, geographical distribution of users and, whenever possible, information/statistics on scientific outcomes (publications, patents, etc.) acknowledging the use of the infrastructure.\\[0.5em]
As indicated in Art. 16.2, the access providers must have the \ac{VA} services assessed periodically by a board composed of international experts in the field, at least half of whom must be independent from the beneficiaries. In the first periodic report, describe how the virtual access providers will comply with this obligation. In the following periodic reports indicate only changes to the existing procedure.\\[0.5em]
When an assessment is scheduled under the reporting period, the assessment report must be submitted as deliverable. 
}

\subsection{Resources used to provide Access to \aclp{RI}}   

\todo{For virtual or trans-national access costs reported as actual costs include, for each access provider, information on how many of the \acp{PM} reported in the use of resources linked to the financial statements have been used to provide access and explain for which task (e.g. scientific support to users, \dots).\\[0.5em]
Information on individual subcontracts must be reported in the use of resources linked to the financial statements in the IT tool. Please mention in the comments field of each subcontract whether it is related to \ac{VA} or \ac{TA}. In addition, all other direct costs items related to virtual or trans-national access must be detailed in the use of resources linked to the financial statements in the IT tool, even if they do not exceed 15\% of personnel costs.}

%%%%%%%%%%%%%%%%%%%%%%%%%%%%%%%%%%%%%%%%%